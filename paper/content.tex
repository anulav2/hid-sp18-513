% status: 100
% chapter: Security

\title{Security Mechanisms in Cloud Computing}

\author{Uma M Kugan}
\affiliation{%
  \institution{Indiana University}
  \streetaddress{107 S. Indiana Avenue}
  \city{Bloomington}
  \state{Indiana}
  \postcode{43017-6221}
}
\email{umakugan@iu.edu}
% The default list of authors is too long for headers}
\renewcommand{\shortauthors}{Uma Kugan}

\begin{abstract}
As we all stepping into the era of \emph{Cloud Computing}, we also have to embrace
the increase in their security concerns. It becomes very vital to find and 
continuously develop new mechanisms to secure the cloud environment. In a 
decentralized architecture, it is extremely difficult to secure the sensitive
data. The computer science industry, alongside of these modern  organizations
need to continue to develop ways to protect their infrastructure, hardware 
and data. This paper is going to highlight the security mechanisms in Cloud
Computing and access control in clouds and challenges related to cloud security.
\end{abstract}

\keywords{Security, Cloud Computing, i524, HID513}

\maketitle

\section{Introduction}
Cloud Computing is the recent buzz word, the new state of the trending 
art technique in the current with almost all organizations are trying
to get into. Cloud computing is regarded as a ``method of running application, 
software and storing the related data in provided computer systems and 
providing customers or other users the access to them through the 
Internet''~\cite{hid-sp18-513-cloud1}. Cloud Computing provides a flexible IT 
architecture which enables companies to minimize their capital expenditure. 
The advantages of Cloud Computing are: reduction of investment in hardware 
and their maintenance, customer can run their business without worrying 
about their software up-gradation and accessibility of information from 
anywhere and their IT services. In Cloud Computing there are many service 
models and deployment models. 

\subsection{Service Model} These services are classified into four main categories: 

\begin{description}

\item[Infrastructure as a Service (IaaS)] - IaaS provides the user
to use storage, processing, networks and other computational resources
that enable user to run and deploy a software. EC2 is the perfect
example~\cite{hid-sp18-513-cloud2}. 

\item[Platform as a Service (PaaS)] - PaaS is a cloud environment or
platform for developers to build cloud applications. PaaS Providers
deliver a database, an operating system, programming language execution
capabilities and web servers. Salesforce.com falls under this 
category~\cite{hid-sp18-513-cloud2}.  

\item[Software as a Service (SaaS)] - SaaS is also referred as
\emph{on demand software} which is a software delivery model in which 
software and associated data are centrally hosted on the cloud. 
Microsoft Office 365 can be operated through browser 
without installing the software on the local machine~\cite{hid-sp18-513-cloud2}. 

\item[Function as a Service (FaaS)] - It allows to deploy function which 
can be invoked through any event. AWS Lambda, Azure Functions, Google 
Cloud Functions are few examples of this category.
\end{description}

\subsection{Deployment Model} There are for deployment models in Cloud
Computing as below.

\begin{description}


\item[Private Cloud]  - The cloud infrastructure is provided for private
use by a single organization including multiple consumers. It may be owned
and managed by the organization and this deployment model is expensive 
when compared to others~\cite{hid-sp18-513-cloud3}.

\item[Community Cloud] - The cloud infrastructure is provided for special use
by a certain community of consumers from organizations that have  common 
concerns. It may be owned, managed, and functioned by more than one 
organization in the community~\cite{hid-sp18-513-cloud3}.

\item[Public Cloud] - The cloud infrastructure is provided for public use by 
the general public. It may be owned, managed, and functioned by any  profitable 
organization, government which provides services to public users. This type 
provides very less security and it cost less~\cite{hid-sp18-513-cloud3}.

\item[Hybrid Cloud] -  The cloud infrastructure is provided for two or more cloud 
infrastructures may be private, community, or public that stay unique  objects, 
but they are linked together by a unified technology or proprietary technology
that enables data and application portability~\cite{hid-sp18-513-cloud3}.

  
\end{description}

\section{Cloud Computing Security}
Cloud Computing security refers to a set of policies, technologies, 
and controls deployed to protect data, applications, and the associated
infrastructure of cloud computing. It is a subset of computer security, 
network security, and, more broadly, information security. Companies that
are planning on migrating to Cloud should be aware that they will be 
surrendering all their private and sensitive information to a third-party
Cloud Service provider which will put the companies to great risk. Hence, 
companies should thoroughly understand and feel comfortable and confident
about the various level of security that the most reliable service provider
can provide to keep companies information and infrastructure 
totally secure.
The data security and privacy protection in cloud computing is similar in 
many aspects to the traditional data security and privacy protection. 
Cloud Computing as a multi-user, multi-tenant distributed environment brings 
the unique security challenges, dependent on the level at which
the user operates~\cite{hid-sp18-513-zissis2012addressing}.

\subsection{Application Level} 
Service Model, Software as a Service (SaaS) falls under this category 
where end client applies to a person or organization who subscribes
to a service offered by a cloud provider and is accountable for use. 
Security requirements are: access control, privacy in multi-tenant environment, 
data protection from exposure (remnants), communication protection, 
software security and service availability.

\subsection{Virtual Level}
Platform as a Service (PaaS) and Infrastructure as a Service (IaaS) falls 
under this where developer moderator applies to a person or organization 
that deploys a software on a cloud infrastructure. 
Security requirements are: access control,application security, data security,
cloud management control security, secure images,virtual cloud protection 
and communication security.

\subsection{Physical Level}
Physical data-center where owner applies to a person or organization 
that owns the infrastructure upon which clouds are deployed. 
Security requirements are: hardware security, legal not abusive 
use of cloud computing, hardware reliability, network protection 
and network resources protection.

\section{Threats, Concerns and Risks of Cloud Computing}
The level of risk in Cloud Computing depends on the type of cloud 
architecture being considered. Security concerns of cloud computing 
fall into two broad categories: security issues faced by cloud providers 
(organizations providing SaaS, PaaS, IaaS via the cloud) and 
security issues faced by their customers (companies or organizations who 
host applications or store data on the cloud). Gartner identifies seven key
areas of cloud security concerns~\cite{hid-sp18-513-gartner}.

\subsection{Privileged User Access}
Sensitive data processed outside the organizations expose the data to the
highest level of risk. So when choosing the cloud service provide, it is
very critical to get enough information on who will have access over the data
that is being stored.

\subsection{Regulatory Compliance}
Company is responsible for their security and integrity of their data irrespective
of where it is stored. Cloud service providers are subjected to external audits
and security certifications~\cite{hid-sp18-513-gartner}. 

\subsection{Data Location} 
Usually cloud service providers may not disclose the location of the data.
It is better to know check with the providers ahead to know their commitment to
obey local privacy requirements on behalf of their customers~\cite{hid-sp18-513-gartner}.

\subsection{Data Segregation} 
Data in the cloud is typically in a shared environment alongside data from 
other customers. Cloud service provider should share all the information
regarding how data is encrypted and how the data is segregated.

\subsection{Recovery} 
It is very critical to know how the data will be recovered and time for restoration
in case of any disaster. 

\subsection{Investigative Support} 
In situation where one needs to investigate, it becomes very difficult as logging 
and data for multiple customers may be co-located.

\subsection{Long-term Viability} 
Data should be available even if the service provider is acquired by some
other larger company.


Based on report of ENISA (The European Network and Information Security Agency),
we can divide the security risks of cloud computing into following three main 
categories~\cite{hid-sp18-513-enisa} :

    Policy and organizational risks: Loss of governance, compliance challenges, 
cloud service termination or failure, cloud provider acquisition.

    Technical risks: Data protection risks, isolation failure, cloud provider 
malicious insider, intercepting data in transit, data leakage, conflict 
between customer procedures and cloud environments.

    Legal risks: These types of risks represent the security risks specific 
for cloud computing, both from the customer and provider point of view. 
We can also distinguish risks not specific for the cloud, such as network 
management, network traffic, privilege escalation, social attacks, and 
natural disasters.

\section{Methods to Secure the Cloud}
Cloud computing and storage provides vital services, poor security and 
privacy in the cloud storage has made many users do not want to 
upload data on the cloud. Both customer and provider are responsible in many
different ways for the security of services. In IaaS, provider has least control
over security as the customer takes care of end o end whereas in SaaS, provider
has full responsibility to secure both the data and systems in the environment.

\subsection{Best Practices for the Cloud Providers}
Cloud providers should ensure, secure and isolate the environment for their 
customers meaning users should gain access only to their environments, data
and their applications~\cite{hid-sp18-513-vmware}. Physical data center should be properly secured 
and access to the building should be allowed only to authorized personnel 
using security keycard, biometric scanning protocols and round-the-clock
interior and exterior monitoring. Each isolated network has to have proper 
perimeter controls and policies to limit access to it.
Host machine protection should include~\cite{hid-sp18-513-diversity} intrusion detection system 
monitoring network and system for any malicious activities and very 
few user accounts as possible with limited administrator access. Provider
should not allow any unnecessary programs running on the machine and should 
perform regular vulnerability scanning of cloud infrastructure to prepare 
for proper mitigation strategies~\cite{hid-sp18-513-winkler}. Provider should enforce strong 
authorization and authentication to provide the customer with secure access 
to their data and resources. Provider should also ensure proper auditing 
mechanisms are in place logging every time the customers or administrators
access and use the resources. Provider should perform frequent backups of 
data and share the details with the customer on measures taken during 
disaster recovery~\cite{hid-sp18-513-vmware}. All API's through which the customers access
the cloud resources with SSL, recommended to provide the secure
communication over Internet has to be encrypted~\cite{hid-sp18-513-diversity}.

\subsection{Best Practices for the Cloud Customers}
Even though a significant amount of security responsibility falls on the 
provider, the cloud's customers have to be aware of certain practices as well.

\subsubsection{Firewalls}
The purpose of a firewall is to prevent the unauthorized access to the local 
computer by third parties via the Internet. Firewalls analyzes the traffic to 
and from the local network and securing there by securing unauthorized traffic.
As firewalls protect a customer's own local network, it is important to ensure
that any local network connectable to the Internet includes strong firewall 
protection. Customers need to understand the two types of firewalls both 
hardware and software. Hardware firewalls are properly configured to 
correctly protect all the machines on a local networks whereas Software 
firewalls have to be installed on individuals machines to prevent a third 
party from taking control of the machine and to protect the customer's 
virtual machines~\cite{hid-sp18-513-diversity}.

\subsubsection{Passwords}
Insecure passwords are considered as the weakest link in the whole
security domain. Customer should enforce users to make sure their 
their passwords are strong, no use of use common dictionary words 
or words associated with the user's personal data, use the passwords
with mix of lowercase, uppercase characters, numbers and special 
characters, use long passwords, change passwords periodically 
and the best use different passwords for different services~\cite{hid-sp18-513-redpaper}.

\subsubsection{Patches and Backups}
Customer has to discuss back up policies with the service provider to be 
certain what is whose responsibility. It is always better to have some 
third-party backup services to have the copies of the data in a case 
of sudden data loss in the cloud services~\cite{hid-sp18-513-diversity}.
Customers should access the impact of installing a local patch on
functioning of their application in cloud.

\subsubsection{Access Control of all Devices Connected to Cloud}
Access to any mobile devices like laptops, mobile phones, tablets have to be 
strictly monitored. Since they devices can be easily stolen, in case of any 
security or data breach, administrators should ideally have the ability 
to remotely wipe any data stored on a computer~\cite{hid-sp18-513-diversity}.

\subsubsection{Virtual Machines Security}
If the customer has signed up for IaaS cloud, where the user sets up
everything on his own including operating system, any middle-ware
and software. It is then the customer responsibility to ensure the
proper security mechanisms are in place~\cite{hid-sp18-513-diversity}.
\subsubsection{Data Encryption}
The data that is shared over the cloud has to encrypted while it 
is being transferred through shared networks. Otherwise there is a 
potential risk of data to be lost or stolen while transferring.

\section{Access control, Authorization and Delegation in Cloud Computing}
Cloud being regarded as a large scale platform for data sharing with 
multiple owners, multiple tenants and users, there is always increase in
demand for fine-grained access control policies. Access control is a policy 
or procedure that allows, denies or restricts access to a system. Most
common models are Mandatory Access Control (MAC), Discretionary Access 
Control (DAC) and Role Based Access Control (RBAC)~\cite{hid-sp18-513-accesscntrl}.

The role-based access control (RBAC) model is the most common 
model because of its simplicity, flexibility in capturing dynamic 
requirements and the support for least privilege principle 
and efficient management~\cite{hid-sp18-513-cloud3}. 

Identity and Access Management (IAM) allows us to control who has access 
to the resources in Cloud Platform project. Resources include Cloud Storage 
buckets and objects stored within buckets. The set of access rules 
applied to a resource is called an IAM policy. 

Access control can be managed with the use of access control lists (ACL).
In Cloud Storage, we can apply ACLs to individual buckets and objects.

The struggle to maintain password security without risking noncompliance
with data security regulations like PCI DSS or HIPAA is becoming more and
more difficult and AAAS protocol comes into play to create safe 
communications among cloud computing tenants in respects to its authority
and priority~\cite{hid-sp18-513-poland}. 

Attribute-based encryption (CPA-BE), is a good cryptography 
method that provide well access control to data. It is a type of public-key 
encryption in which the secret key of a user and the ciphertext are 
dependent upon attributes. For instance, ciphertext are dependent on attributes 
such as the country in which he lives, or the kind of subscription he has.
In such a system, the decryption of a ciphertext is possible only if the set 
of attributes of the user key matches the attributes of the ciphertext~\cite{hid-sp18-513-poland}.

Security Assertion Markup Language (SAML), Extensible Access Control Markup Language
(XACML) and web services standards helps to ensure cross-domain accesses are
properly specified and enforced.

Role mining uses the already existing system configuration data to define the roles. 
In a cloud, users acquire different roles from different domains based on the 
services they need. To define global policies, RBAC type systems configurations 
can be used in conjunction with IAM and ACL to define the global roles and 
policies.

\section{Conclusions}
Cloud computing definitely has lots of advantages such as cost effective, 
unlimited storage capacity, much easier backup and recovery, automatic
software integration, easy access to information and faster deployment. 
However, flexibility has paved a way for number of security issues.
The bigger the clouds are, the bigger the risks and responsibilities are. 
There exists a much greater need for proper security measures.
Cloud Computing security lies in the hands of both the consumers and their
service providers and both of them working together can significantly 
develop more secure way of delivering cloud computing than traditional 
approaches. In this paper, we presented guidance on cloud computing security,
their challenges and their best practices for both provides and consumers.

\begin{acks}
The author would like to thank Dr. Gregor von Laszewski for his support and 
suggestions in writing this paper.
\end{acks}

\bibliographystyle{ACM-Reference-Format}
\bibliography{report} 
