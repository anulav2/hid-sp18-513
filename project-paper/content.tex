% status: 98
% chapter: Security

% status: 0
% chapter: Security

\title{Service for Managing the Public Key}

\author{Uma M Kugan}
\affiliation{%
  \institution{Indiana University}
  \streetaddress{107 S. Indiana Avenue}
  \city{Bloomington}
  \state{Indiana}
  \postcode{43017-6221}
}
\email{umakugan@iu.edu}
% The default list of authors is too long for headers}

\renewcommand{\shortauthors}{Uma Kugan}
\begin{abstract}
SSH keys are used to control access to any system using public and private
key pair. The user creates the private keys and store it securely on his 
machine and shares the public key. Usually IT admins distributes and
manage SSH keys which creates more hassle for the admins to manage.
In this paper, we are going to create a rest service to manage their
public key.

\end{abstract}

\keywords{hid-sp18-513, Security, Key Management}


\maketitle

\section{Introduction}
SSH is a way for anyone to connect to any servers in a more secured way. 
Any information that is exchanged between host computer to the server is encrypted
which prevents someone from snooping the data. There are two ways by which user
can authenticate to connect to the servers : one by providing user name and 
password combination or by using SSH keys.

SSH keys are basically files that are generated by the OpenSSH program on any 
computer. An SSH Key consists of two parts. A public key and a private key. 

\subsection{Private and Public Key}

The private key file is kept hidden on computer in a specific directory 
which can be accessed when connecting to a server. The public key is placed 
on the server that we are trying to get access to. Now, when we try to 
connect to the server, computer presents information to the server via the 
SSH key that proves to the server its really you trying to get access. 
Without the public SSH key on the server and private key on the computer
nobody can access your account using SSH.

SSH keys are used in most organizations. The problem with SSH keys is keeping 
track of whose SSH key has been placed on which machines and making sure that 
the keys get changed every couple of days/weeks so that it is hard for anyone 
to guess what the key Is and try to attack the server by forging your identity.
One of the main issues is that IT admins have to spend a lot of time visiting 
each server, changing out the private keys then go back to each user and have 
them make changes on their laptop. The rest api provides the easier solution to
exchange keys between the servers ~\cite{hid-sp18-513-ylonen2013user}.

\section{How SSH Works}
SSH uses a client-server model of computing, allowing users to establish
secure communications between two hosts, whether UNIX, Windows
or virtually anything else. SSH is also increasingly important as the use
of cloud computing grows, helping organizations limit exposure while
connecting to a cloud-based virtual machine over the Internet by providing
a local gateway-style system via an SSH endpoint.
The server enables incoming SSH connections to a host and handles user
authentication, authorization and related tasks. The client connects to an
SSH server and makes requests after authenticating to the server. 
An SSH session is the ongoing connection between a client and a server begins
after the client successfully authenticates to a server and ends when the
connection terminates. Once the system authenticates a user, one or more
channels may be opened within the connection, with each channel acting
as an individual data link and separate pathway for information
~\cite{hid-sp18-513-sans}.


\section{Role of Secure Shell}
The Secure Shell (SSH) protocol, developed in the mid-1990s, and it  
used everywhere both in traditional and virtual environments to 
authenticate Unix users and their applications to remote, 
internal systems and encrypt the resulting session traffic. 
In order to authenticate, an user or an application must share
its private SSH key file to a target system that possesses its 
corresponding public SSH key file. 
Once the key pair is validated, the user or application is granted 
access to the protected account. In effect, this means that any user
or application with access to a private key may access any target 
system such as Unix and Linux systems, virtual machines, network 
devices, or file transfer solutions that contains the corresponding
public key.
In a one-to-many environment, in which one private SSH key is used 
to attain privileged access to many target systems, organizations 
can no longer afford to ignore these privileged private keys. 
To protect the heart of the enterprise,organizations must 
proactively secure and manage all of their privileged account 
credentials, including both passwords and SSH keys, to achieve
consistent, comprehensive protection against advanced 
attacks~\cite{hid-sp18-513-cyberark}

\section{SSH Keys: The Hidden Credentials} 
SSH keys are always the major concern to security teams because 
these credentials can be easily created, and are then difficult to
track, manage or control. In any environment without any proper
controls in place, any user with access to any machine can generate
an SSH key pair that will forever grant that user direct access to
the system. And, because there is no built-in oversight to SSH keys,
no one may ever know. Further, because SSH is commonly used in 
automated application to application authentication,
SSH key pairs can be generated, distributed and never thought of again, 
leaving applications and application servers vulnerable to attacks 
using unmanaged, outdated SSH keys~\cite{hid-sp18-513-cyberark}. 
In the enterprise environment with hundreds of IT users and 
thousands of systems managing the SSH challenge becomes exponential. 
A large enterprise may have thousands or even millions of valid 
SSH key pairs, many of which may no longer be needed by authorized 
users or applications. As a result, attackers can exploit this 
vulnerability and gain privileged access to critical systems. 
Without central management or knowledge of who 
is accessing what, organizations cannot see where their risks and
vulnerabilities lie much less take action to address them
~\cite{hid-sp18-513-cyberark}. 

\section{SSH Keys Vulnerabilities}
Vulnerabilities with SSH range from weaknesses in the protocol
itself to configuration, implementation and management 
complexities that can lead to dangerous mistakes.
Poor SSH key management practices can result in compromise 
of administrative privileges, expanded breadth of attack surfaces
and other issues that could easily be avoided by proper training
and more effective processes backed by automation and tools
~\cite{hid-sp18-513-sans}.

\subsection{Configuration Issue}
Application developers must learn how to keep both configuration
and private key files secure. With these files, a malicious 
individual can easily impersonate an authorized
user (or host) and easily connect to a remote host and application.
Before deploying any new devices in a networked environment,
it is always the best practice to change default passwords
for all applications, operating systems, routers, firewalls
and all other systems~\cite{hid-sp18-513-sans}.

\subsection{Gaining access with SSH Key}
Lack of defined governance for SSH key-based trust 
relationships can allow an attacker who compromises one 
system to quickly pivot from one system to another and extend
a breach into other parts of an organization. Enough keys may be 
stolen, leaked or disused—without having terminated their trust
relationships—to pose a serious, ongoing threat to an organization.
Attackers once get access to these keys, they will generates a new 
key pair and adds the new authorized key to authorized keys file
which is not usually monitored or audited.
The December 2014 Sony hack included the leak of SSH keys as well
as password lists, which led to the compromise of related services and
accounts; mishandled SSH keys may have even facilitated the initial 
compromise~\cite{hid-sp18-513-sans}.


\section{How to Manage SSH Risks?}
To reduce the threat and comply with regulatory requirements, 
organizations should take a proactive, end-to-end approach to SSH key
security and management. Such an approach should include proactive
controls to secure, manage and monitor the creation, storage and use
of all privileged SSH keys. By employing a layered approach that includes
the critical security measures outlined below, organizations can build
out a proactive SSH key management program that reduces risks and helps 
meet regulatory requirements.

\subsection{Secure key storage}
To reduce the risk of stolen SSH keys, organizations should store 
private user and application keys in a highly-secure
centralized repository that supports strong access controls.
In an unmanaged state, SSH keys are stored as system files and can 
easily be moved or copied. As a result, critical systems are only
as secure as the machines or devices on which the private keys are
stored. To mitigate this risk, organizations should consider removing
SSH keys from vulnerable endpoints and systems and instead storing 
them in a highly secure, highly available central repository that 
supports access controls such as automated workflows for 
elevated-privilege requests and strong authentication to quickly 
verify user and application identities. With a secure centralized 
model, organizations can remove critical credentials from individual
machines, protect all credentials equally and centrally track all usage
of SSH keys by users and applications~\cite{hid-sp18-513-gutmann}.
\subsection{Proactive key rotation}
Similar to static passwords, static SSH keys pose an on-going risk
to organizations. Compromised private SSH keys can provide unauthorized
users with permanent backdoor access into critical systems. 
To mitigate this risk, organizations should rotate all SSH key 
pairs at regular intervals, just as they rotate privileged passwords
today. In unmanaged environments, with keys dispersed across devices,
the prospect of frequently rotating thousands of keys can be daunting.
However, once all user and application SSH keys are consolidated under
one central management system, key pairs can be automatically rotated
and public keys can be automatically distributed to target systems. 
With central management and automated key rotation, organizations can
better secure SSH keys and the target systems they protect, 
implement best practices and comply with regulations without burdening
the IT team\cite{hid-sp18-513-gutmann}.
\subsection{Privileged session monitoring}
Proactive protection of privileged credentials is a critical element
in privileged account security, but proactive controls must be 
complemented with equally strong detection and response capabilities.
To minimize any potential damage caused by advanced external and
inside attackers, organizations should proactively monitor all
privileged sessions, including those that occur via SSH. When abnormal
activity is detected, security teams should have the ability to
remotely terminate the suspicious session to disrupt the potential attack.
Similarly, organizations must also record all privileged session activity.
When an incident is detected, response teams require immediate access to
session recordings and detailed audit logs to determine exactly what
happened and what steps must be taken to re-mediate the incident.
Access to these privileged session recordings and audit logs can also be
granted to auditors to help prove compliance with regulations
~\cite{hid-sp18-513-gutmann}.

\section{Measures to be taken for securing SSH Keys}
Both internal and external auditors must add Secure Shell key 
scanning and management to their checks. Proper controls and tools
must be put in place for managing Secure Shell keys. The real issue
is authorized keys, as they are the ones that grant access. No
matter how much you try to protect private keys, it is of no help 
until the millions of existing authorized keys have been sorted out.
Regulators must establish firm deadlines for enterprises and execute 
the revoking of all access when no longer needed. However, 
implementation and audit guidelines should be clarified
to ensure Secure Shell keys are taken into account. Boards, 
audit committees, Security and Governance and risk management 
officers must ensure Secure Shell key-based access is
properly accounted for in their organizations to avoid civil 
and criminal liability~\cite{hid-sp18-513-network}.

\section{NIST ISSUES GUIDANCE ON SSH KEY MANAGEMENT}
US National Instute of Standards and Technology (NIST) as issued guidance on
SSH key management as NIST IR 7966. 

\subsection{REGULATORY COMPLIANCE REQUIRES SSH KEY MANAGEMENT}
Typical requirements for compliance include:
\begin{itemize}
	\item Managing identities and credentials - SSH keys 
	are access credentials
	\item Provisioning and termination process for access 
    - including access based on SSH keys
	\item Segregation of duties - elimination of 
	key-based access from test and development systems into
	production
    \item Disaster recovery - limiting attack spread
    from primary systems to disaster recovery sites and
    backup systems
    \item Privileged access controls - SSH keys are often used to 
    bypass jump servers
    \item Boundary definition and documentation of connections for 
    systems that contains sensitive data such as payment systems,
    financial data environments, patient data 
    environments, or between government information systems
    \item Incident response and recovery
\end{itemize}
~\cite{hid-sp18-513-sshkey}.

\section{Deployment}
\subsection{Setup}
\paragraph{Git repository}
Setup the git repository:
\begin{verbatim}
export HID=hid-sp18-513 
mkdir -p ~/github/cloudmesh-community
cd ~/github/cloudmesh-community 
git clone
https://github.com/cloudmesh-community/$HID.git
/project-code
\end{verbatim}


Validate code has been cloned into the directory.


Technology : Python, Rest API, Sqlite3

Install \textbf{Python 2.7.13} via PIP

Install \textbf{pip install flask-restful} 

Database : sshkeymgmt.db and three tables have been created
to  capture access request, validation and status of the request.

User will call the API requestaccess to request for sharing the key
from one server to other server with host name and user id of the 
source and target and also the location of public key in source host.

API validate and action will validate if the user have a login in the
target server. If not the request will be declined and the status will be
updated on the table.

Once the request is approved, the third API will actually copy the keys
to target server. Before it copies, API will validate if the user calling the
API have access to target server or whether he is any group and or has
admin access.

\section{Conclusion}
SSH is an widely used important protocol that provides encrypted, 
authenticated communications in a variety of configurations. Public key
authentication is very vital to any organization, as it automates 
access and can easily provide ease of login to accounts with elevated 
privileges. However, SSH is also especially vulnerable to complexities
that involve configuration, implementation and management. In order to
reduce risk, proactive monitoring and implementation of best practices 
both during implementation and afterward is needed. 

In a big organizations, there may be huge volumes of keys in use and
remediation of an environment where SSH keys and trust relationships 
have never been fully managed is complex, labor-intensive and 
time-consuming. But there are frameworks available in the industry which 
can be used or the project has few basic API to manage the key in smaller 
environment which can be further enhanced to larger scale.

\begin{acks}
The author would like to thank Dr. Gregor von Laszewski for his support and 
suggestions in writing this report and succesfully completing the project.
\end{acks}

\bibliographystyle{ACM-Reference-Format}
\bibliography{report}







