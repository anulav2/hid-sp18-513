% status: 2
% chapter: Security

\documentclass[sigconf]{acmart}

\begin{document}
\title{Service for Managing the Public Key}

\author{Uma M Kugan}
\affiliation{%
  \institution{Indiana University}
  \streetaddress{107 S. Indiana Avenue}
  \city{Bloomington}
  \state{Indiana}
  \postcode{43017-6221}
}
\email{umakugan@iu.edu}
% The default list of authors is too long for headers}

\renewcommand{\shortauthors}{Uma Kugan}
\begin{abstract}
SSH keys are used to control access to any system using public and private
key pair. The user creates the private keys and store it securely on his 
machine and shares the public key. Usually IT admins distributes and
manage SSH keys which creates more hassle for the admins to manage.
In this paper, we are going to create a rest service to manage their
public key.

\end{abstract}

\keywords{hid-sp18-513, Security, Key Management}


\maketitle

\section{Introduction}
SSH is a way for anyone to connect to any servers in a more secured way. 
Any information that is exchanged between host computer to the server is encrypted
which prevents someone from snooping the data. There are two ways by which user
can authenticate to connect to the servers : one by providing user name and 
password combination or by using SSH keys.

SSH keys are basically files that are generated by the OpenSSH program on any 
computer. An SSH Key consists of two parts. A public key and a private key. 

\subsection{Private and Public Key}

The private key file is kept hidden on computer in a specific directory 
which can be accessed when connecting to a server. The public key is placed 
on the server that we are trying to get access to. Now, when we try to 
connect to the server, computer presents information to the server via the 
SSH key that proves to the server its really you trying to get access. 
Without the public SSH key on the server and private key on the computer
nobody can access your account using SSH.

SSH keys are used in most organizations. The problem with SSH keys is keeping 
track of whose SSH key has been placed on which machines and making sure that 
the keys get changed every couple of days/weeks so that it is hard for anyone 
to guess what the key Is and try to attack the server by forging your identity.
One of the main issues is that IT admins have to spend a lot of time visiting 
each server, changing out the private keys then go back to each user and have 
them make changes on their laptop. The rest api provides the easier solution to
exchange keys between the servers.

Technology : Python, Swagger, Rest API

To add users who do not have SSH keys, generate a new SSH key for each new user.

To add users who have existing SSH keys, locate the public SSH key file for each user.

Once public key is located, we should add those key to authorized keys 
to target server.

There are two options :

1. Login should exist in both the server - straight forward as the user exist 
         in both source and target server.

2. If the login does not exist,
        - We need an api to get username, host, port and pub key from source
            and user, host & port for target
        
3. Root will create login & add pub keys to authorized keys
	- Api to post pub keys from source to target

4. Before we post the keys, we need to ensure the source:user have access that
   We can either maintain two config file (or manage thru mongodb) for now with the following info :

  Config File (i)  Group   Dir        Access level

  Config File (ii) Groups    Members
 
If they have access, then we will add their keys.

Usually when we ssh for first time, it will ask for password, so user should have the credentials for both
the server.

If the user from source wants to ssh to some other id on target, he can login only after root adds the key.


\section{Conclusion}

Put here an conclusion. Conclusion and abstracts must not have any
citations in the section.


\begin{acks}
The author would like to thank Dr. Gregor von Laszewski for his support and 
suggestions in writing this paper.
\end{acks}

\bibliographystyle{ACM-Reference-Format}
\bibliography{report}

\end{document}



